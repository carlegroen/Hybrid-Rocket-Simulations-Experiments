% !TEX root = main.tex
\chapter{Introduction}

It's not rocket science.

The following report contains simulations and observations of the hybrid rocket MEOWTH II (\textbf{M}echanical \textbf{E}ngineering \textbf{O}bservational \textbf{W}ingless \textbf{T}utoring \textbf{H}ybrid rocket) created by \emph{Team Rocket} \cite{bulba} at Navitas, Aarhus University. The first rocket was built in 2014 by a previous group of students, but has since been passed along to other students, with the intent of teaching the basics of rocket science.


During the previous MEOWTH's tests, a large pressure spike during the startup phase has been observed. This can be seen in the old video here: \url{bit.ly/1SYp1Ug} in the timespan of seconds 5-7. From the video, it appears that just as the rocket exhaust reaches supersonic speeds, \emph{something} explodes, extinguishing the flame. We know it happens before the rocket goes supersonic due to the lack of shock diamonds prior to the event \cite{rockProp}.


This bachelor thesis attempts to simulate the rocket's startup condition in order to explain and understand the event. My advisor, Gorm Bruun Andresen, proposed a hypothesis that the pressure spike occurs at the point where mass-flow is choked in the throat. As the flow is sub-sonic, the amount of matter exiting through the nozzle is gradually increasing. When the flow goes sonic or beyond, it is suddenly capped. This abrupt change in physical properties may be able to explain this event, and the following contains the theories and thoughts that went into simulating the rocket engine's startup.


In order to create a safer rocket with more measurement options, MEOWTH II was built. Gorm Bruun Andresen, Alex Nørgaard, a fellow student, and I personally assembled the rocket. The rocket can be seen on the frontpage and on figure \ref{fig:rocketpic}, and each section contains at least four possible points measurements for various instruments.


The purpose of these simulations are ultimately to improve the rocket and rid it of large pressure variations. To see improvements, it is necessary to measure how the rocket behaves under different conditions. Therefore, setup and calibration of various measurement equipment is essential in order to retrieve excellent data. This is also covered in this report, as the experimental part is of equal importance. Measuring the rocket chamber's temperature is of great inconvenience, as the several thousands of kelvin are too much for even the strongest of thermocouples \cite{thermocoup}. This issue and its solution is discussed thoroughly in the sections below.

All measurements and tests could not have been carried out if not for the test facilities provided by Peter Madsen of Raketmadsen's spacelaboratory. MEOWTH II's tests were carried out on the 3rd to the 4th of May, 2016 in Copenhagen.

The entire project is publicly available on \url{https://github.com/carlegroen/bachelors_degree}, where all work files, data and various notes can be found.
