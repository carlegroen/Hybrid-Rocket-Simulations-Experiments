% !TEX root = main.tex
\chapter{Rocket tests}

Testing of the rocket engine ensued at "Raketmadsens Rumlaboratorium" in Copenhagen on the 3rd to the 4th of may 2016. The tests were carried out by me in company by my advisor Gorm Bruun Andresen, and seven other students from Navitas.

Day 1:

Rocket setup, establishing connection to remote pc, being locked out, installing labview

Day 2:

Worthy mentions: Tighten flanges correctly, make sure O-rings are in place.


\section{Firing Procedure}
Launching the rocket requires several crucial steps in order to safely ignite the engine. Safety is the number one priority, thus, a stepwise checklist is necessary.

PRELAUNCH
\begin{enumerate}
  \itemsep0em
  \item Insert grain
  \item Insert foam permeated with \chem{KMnO_4}
  \item Assemble rocket chambers
  \item Establish remote access to control computer
  \item Ensure measurement options are correct
  \item Check signal and restart ManuWare
  \item Create new data-log file
\end{enumerate}
ALL CLEAR AREA EXCEPT FUEL RESPONSIBLE PERSON
\begin{enumerate}
  \itemsep0em
  \item Equip \chem{H_2 O_2} safety equipment
  \item Fill tank with \chem{H_2 O_2}
  \item Pressurize tank
  \item CLEAR THE AREA
  \item Start data-collection and cameras
  \item Arm the rocket
  \item Fire the rocket
  \item Stop data-collection and cameras
  \item Remove external pressure compressor
  \item Depressurize tank
  \item Area is safe
\end{enumerate}

Peter Madsen and his assistant Stefan Eisenknappl are the only two people present when loading the rocket with \chem{H_2 O_2}. The procedure is executed from start to end at each launch, and has several areas where it can be improved if time permits.

The control computer should be replaced by a microcontroller, such as an arduino or raspberry pi. As of this project, the data bandwidth is too low in either alternative, but it is a viable solution in the near future.

In order to reduce reload time, adding more piping to the christmas tree is necessary, as air leaving the \chem{H_2 O_2} tank flows back, spilling the \chem{H_2 O_2} concentration out of the funnel. When a suitable final design is done, welding the pieces together is a superior alternative to using bolts and nuts. Welding removes any chance hydrogenperoxide leaking, damaging the rocket and measurement equipment.

The rocket's arming and controlpad needs to be set up in a smarter, more convenient way. The ideal setup is to have a single controlbox, that when starts data-collection as soon as the rocket is armed, and notes when it is fired. This would allow everything to be controlled from a single box, with measurements being saved to a raspberry pi situated at the base of the rocket. Data can then instantaneously be read from a secondary computer through cable or wireless connection. As the rocket's final destination is space, continuously improving data-transfer and making the rocket an individual unit is paramount.
