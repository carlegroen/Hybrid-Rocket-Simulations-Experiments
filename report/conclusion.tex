% !TEX root = main.tex
\chapter{Conclusion}

From my simulation, I propose the hypothesis that the time it takes for the fuel and oxidizer to reach autoignition temperatures is the culprit of the pressure spikes observed at ignition. The proposed hypothesis is that the lower the autoignition temperature, the lower the pressure spike. Therefore, adding a preburner fuel such as ethanol or something with a low autoignition point should reduce this spike greatly. A pilotflame might be the solution to this problem as well, as piloted ignition occurs at lower temperatures, expending the available oxygen faster.

A proposed test experiment could be to add silane (\chem{H_4 Si}), white phosphorous (\chem{P}) or carbon disulfide (\chem{CS_2}), all which have very low (\SIlist{21;34;90}{\celsius}) autoignition points. These may work suitably as a self-igniting pilot flame, until the MDF's autoignition point is reached.
Alternatively, an "anti--test" can check the hypothesis: Increase the grain's autoignition temperature, in order to see how it behaves in the opposite limit. By increasing the autoignition temperature, a higher pressure--spike should be observed. Furthermore, pre--heating the grain to \emph{just} below its autoignition temperature using electrical heating would allow almost instant combustion as soon as oxygen reaches the heated surface. This should could quite easily be tested and controlled, which makes it a possible project for the next group of rocketeers.
