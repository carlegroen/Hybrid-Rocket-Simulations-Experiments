% !TEX root = main.tex
\chapter{The Project's Future}

After intitial combustion, the resulting shockwave may move faster than the flame velocity, thus quickly extinguishing the flame before resuming a steady state. This may be an explanation for the small dip in observed pressure around the initial spike seen in the test. Thus, this is someone worthwhile calculating if moving on further with the project. The solution is probably contained in the book \emph{The Principles of Fire Behaviour}, as found in the bibliography \cite{principlesoffire}.

Simulating the change in heat capacity is also essential in order to get a more precise result. Therefore, this is an obvious place to expand on the already established code.
