\chapter{Theory}

  Understanding the theory behind the rocket's flow requires a basic knowledge on rockets. Therefore, the first theoretical segment concerns basic rocketry, followed by a more advanced segment of nozzle theory.

\section{Basic Rocket Science}

  A rocket engine consists of a few fundamental elements. A rocket engine is a type of jet engine that, in contrast to duct jets, carry their own rocket propellant. Jet engines as seen in aeroplanes are usually situated with a duct, confining the air flow. Rocket engines on the other hand carry a supply of oxygen and rocket propellant, which allows them to function even in vacuum.


  Rocket engines work by obtaining thrust in accordance with Newton's third law. The internal combustion chamber accelerates fluids through a propelling nozzle to high speeds. The fluid is most often a gas created from mixing fuel and oxidizing components in a the combustion chamber. The exhaust is accelerated to supersonic speeds by expansion in the nozzle, which forces the engine in the opposite direction.

  Most rockets used today are liquid rockets which store their propellant and oxidizing component in separate tanks. The liquid fuel is then forced into the combustion chamber for consumption. Solid-fuel rockets contain propellant prepared with a fixed fuel and oxidizing component. The fuel is called 'grain', and the storage compartment for the grain is the combustion chamber. A hybrid rocket is the mixture between the two. Most often, hybrid rockets contain a solid fuel, or 'grain' and liquid or gaseous oxygen, thus earning the name hybrid engine. Variations of this engine type do exist, but this configuration is the most often used \cite[chapter 16, p.~605]{rockProp}. Solid oxidizers are uncommon as they are problematic and have worse performance than liquid oxidizers.

  Liquid and hybrid engines both use injectors to disperse oxygen and propellant into the combustion chamber. For a hybrid engines, this means spreading oxygen to the grains surface to allow combustion.

  Hybrid rockets are inherently safer than its two counterparts, and accidents are less volatile as accidental fuel mixing is a non-issue. The oxidizer and fuel are almost always contained in separate chambers, which also reduces the mechanical complexity of the rocket in comparison to liquid rockets.

\section{Nozzle Theory}

  The nozzle consists of three parts: The combustion chamber, the converging into a throat, and diverging section after the throat. A rocket's effectivity is highly dependent on the shape, size and ratios between these three segments. Accordingly, it is imperative to study these parts of the rocket's design.

\subsection{Combustion Chamber}
\subsection{Throat}
\subsection{Nozzle}
